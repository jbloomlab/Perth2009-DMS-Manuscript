\documentclass[11pt, oneside]{article}   	% use "amsart" instead of "article" for AMSLaTeX format
\usepackage{geometry}                		% See geometry.pdf to learn the layout options. There are lots.
\geometry{letterpaper}                   		% ... or a4paper or a5paper or ... 
\usepackage{color}
\usepackage[parfill]{parskip}    		% Activate to begin paragraphs with an empty line rather than an indent
\usepackage{graphicx}				% Use pdf, png, jpg, or eps§ with pdflatex; use eps in DVI mode
								% TeX will automatically convert eps --> pdf in pdflatex	
								
												
\usepackage{amssymb}
\usepackage{hyperref}

\newcommand{\comment}[1]{{\color{red}[\textsl{#1}]}}
\newcommand{\response}[1]{{\color{black}#1}}

\title{Response to reviews of ``Deep mutational scanning of hemagglutinin helps predict evolutionary fates of human H3N2 influenza variants'' for \textit{PNAS}}
\author{Juhye M. Lee, John Huddleston, Michael B. Doud, Kathryn A. Hooper,\\Nicholas C. Wu, Trevor Bedford, Jesse D. Bloom}

\begin{document}
\maketitle

We thank the reviewers for the careful reading of our manuscript.
We were gratified that they recognized the quality of our work: one of the reviewers said, ``the experimental setup is sound and there is strong correlation among technical and biological replicates.''

The main contribution of our work is to use show that deep mutational scanning of an H3 HA provides information that is informative about the evolutionary fates of H3N2 influenza virus variants in nature.
We appreciated the reviewers' recognition of the important step our work takes in linking experiments with quantitative models of influenza virus evolution, and their recognition that we carefully discuss the caveats as well as the strengths of our approach.
Reviewer \#1 did make an important suggestion to look at HA sequences only from unpassaged virus isolates---in the revised version we have done that, and found that our data are even more predictive of evolutionary fate for these higher quality sequences.

Most of the reviewer questions had to do with more tangential aspects of our work, such as validation of specific mutations and phylogenetic analyses.
In the revised version, we have addressed all of these questions.
Therefore, we think that you will find that the revisions have further strengthened the paper.

\emph{Below, the reviewer comments {\color{blue} are in blue}, and our responses are in black.}

\color{blue}

\subsection*{Reviewer \#1 Comments}

\subsubsection*{Comments:} 
In their manuscript Lee et al. examine the tolerance of an H3 virus HA for single point mutations using deep mutational scanning. Using the data obtained from cell culture experiments the authors then try to understand if there is a correlation between the observed changes in vitro and actual virus evolution in vivo. Finally, they test if there is any correlation between H3 and H1 mutational tolerance and basically find that measurements made with H1 HA are not helpful to predict H3 evolution. The manuscript is written in a relatively confusing way and the experimental setup is problematic. Some of the obtained data also need re-evaluation/experimental validation. 

\response{We are sorry that the reviewer found some parts of the manuscript confusing.
We have revised the parts of the text that the reviewer highlighted, performed additional validation experiments, and clarified the experimental setup.
These modifications should fully address the reviewer's concerns.}

\subsubsection*{Major points} 

1) It is disturbing that there is no preference for having a methionine in the first position of the HA ORF. The authors discuss this and refer to one 2002 paper that shows additional start sites for non-human H2. This is a pretty weak explanation. If this is really the case the authors should actually isolate/rescue these viruses and test their fitness relative to wild type. 

\response{
We have now used reverse genetics to generate viruses carrying a mutation of the initiating methionine to lysine.
These start-codon mutant viruses grow to appreciable titers, suggesting that there is indeed some flexibility in the site of translation initiation in this HA.
These data are in Figure S4 of the revised manuscript.
As we note in the manuscript, this finding is concordant with prior work on suggesting additional translation initiation sites sites in H2 HAs.
Because translation initiation of HA is not the focus of our manuscript, we have not investigated the exact mechanism or identity of the alternate start sites.
However, the data we have added clearly show that the canonical ATG is not strictly necessary for viral growth.
}

Of note, position 1 for the H1 HA in the supplementary figure is not even shown. 

\response{
The reviewer is correct. 
Figure~S3 shows data for a H1 HA as measured previously by Doud and Bloom (2016), and these prior data are simply presented here in H3 numbering for comparative purposes (the original Doud and Bloom paper used a different HA numbering scheme that makes site-to-site comparisons with the current study difficult). 
The Doud and Bloom study used mutant libraries in which all codons in HA \emph{except} for the one encoding the N-terminal methionine were mutagenized.
Therefore, that study did not estimate the effects of mutations to the first codon, which is why it is not shown in Figure~S3.
We have added a sentence to the legend for this supplementary figure explaining this fact.
}

2) While other important features of the HA like several cysteines seem to be conserved, there is no pressure on other elements which are considered important for viral fitness including the NGT glycosylation site in the beginning of HA1 which is conserved in all group 2 HAs or the adjacent NAT glycosylation site which is conserved in H3, H7, H10 and H15. On the other hand, head glycosylation sites, which are highly variable in nature, seem to be conserved. Again, viruses without these glycosylation sites should be rescued and evaluated for fitness relative to wild type. 

\response{We have now also individually evaluated the reverse-genetics titers of three glycosylation site mutants.
Two of the mutants, T24F and T40V, remove the aforementioned highly conserved but mutationally tolerant NGT and NAT glycosylation sites near the start of HA1.
The third mutation, S287A, removes a glycosylation site that our experiments suggest is under functional constraint.
The new Figure S5 shows that the T24F and T40V mutants grow to titers comparable to wildtype, supporting the conclusion that they are expendable at least in cell culture. 
However, the titer of the S287A mutant is modestly lower than wildtype, supporting the conclusion that this glycan contributes to viral growth in cell culture.

One possible explanation is simply that cell culture does not fully capture the constraints on HA in nature.
But a more interesting possibility is that conservation in nature does not strictly parallel mutational tolerance.
The tendency for a site to change in nature depends on how much selection there is for the mutations.
For instance, Wei et al (\textit{Science Translational Medicine}, 2:24ra21, 2010) have pointed out that glycosylation motifs evolved rapidly in the globular head of HA from human H1 influenza in the period from 1918 to 2008 (seasonal H1N1), but changed very little in the same period in the swine H1 lineages descended from the 1918 virus.
The likely reason is differential strengths of selection: human influenza is under strong pressure from accumulated immune memory, whereas swine influenza is under less immune pressure because pigs are short lived and so are often naive when infected with influenza.
This fact highlights that even mutationally tolerant sites will only change rapidly if there is a benefit to doing so, while even somewhat constrained sites can change if there is sufficient pressure.
A strength of our experiments is that we have controlled the selection pressure (selection only for viral growth), and so can measure mutational tolerance in the absence of the confounding effects of immunity.

In addition to the data in Figure S5, we have added the following text to the Results:

\begin{quote}
\textsl{
The disparity between the relative conservation of these sites in nature and their mutational tolerance in our study could be because cell culture does not fully capture the constraints on HA function in nature, or could be because these sites are not under strong immune pressure and so mutations at them are not positively selected in nature.}
\end{quote}
}

3) It is also surprising that the transmembrane domain seems to be highly plastic, a feature that is certainly not present in nature where it is highly conserved within H3 HAs. 

\response{We have validated that one of these mutations in the transmembrane domain, C199(HA2)K, reaches reverse-genetics titers comparable to wildtype.
These data are in the new Figure S6.
In addition, various other studies have found that chimeric HAs with transmembrane domains from completely different proteins can still maintain HA cell surface expression and fusion activity (Melikyan et al 1999 Mol Biol Cell, Schroth-Diez et al 1998 JVI, Dong et al 1992 JVI, Roth et al 1986 JCB).

We posit that the disparity between the conservation of the transmembrane domain in nature and its apparent mutational tolerance in our experiments could arise at least in part from a lack of immune pressure for mutations in nature, as discussed in the response to the previous point.
}

4) Were sequences from passaged viruses (isolated on MDCKs and other cell lines) included in the comparison between in vitro mutagenesis data and natural evolution data? They need to be excluded since their presence might induce a bias towards what is tolerated/preferred in cell culture. Only sequences obtained directly from patient specimens should be used for the comparison. 

\response{
This is an excellent question.
We now perform a separate analysis of the relationship between our measured mutational effects and maximum frequency in nature for mutations from unpassaged isolates only.
Because sequences from unpassaged isolates have only become common relatively recently, we could perform this analysis only for the post-Perth/2009 partition of the H3N2 HA phylogenetic tree.
The new Figure 5 shows that the correlation between our measurements and natural evolution is actually \emph{stronger} when we look only at unpassaged isolates.
This makes sense---lab passage mutations will not do well in nature, but will be scored favorably in our experiments.
Therefore, it makes sense that our approach works \emph{even better} when the analysis is restricted to unpassaged isolates.
We thank the reviewer for this suggestion, which has strengthened our work.
}

5) It is unclear how the ``rescaling'' is performed and what it exactly does. The authors need to described and justify this better and should also provide the non-rescaled data in a ``sequence logo'' figure for comparison. 

\response{
The re-scaling of amino-acid preferences by a stringency parameter is a well-defined mathematical procedure that is explained in detail in Hilton et al (\textit{PeerJ}, 5:e3657, 2017), which is cited in the manuscript.
This re-scaling procedure has now been used fairly widely in the literature. 
Importantly, re-scaling does \emph{not} affect any of the evolutionary analyses performed in this paper since it just corresponds to a constant that multiplies all mutational effects (e.g., a change in units on the x-axis of Figure~5).
Specifically, if the re-scaling parameter is $\beta$, then the mutational effects calculated using Equation~1 in the manuscript are just multiplied by $\beta$ after re-scaling.
Therefore, for analyses of mutational effects, the re-scaling simply corresponds to a change in units (e.g., comparable to showing an antibody titration in ng/mL versus $\mu$M), and therefore has no effect on the conclusions.
We have added the following text in the Methods explaining this fact:

\begin{quote}
\textsl{The amino-acid preferences were re-scaled by the stringency parameter using the approach described in Hilton et al (2017).
Note that the re-scaling simply puts the amino-acid preferences on a useful scale for visualization in logo plots, and has no effect on any of the quantitative conclusions relating the deep mutational scanning to natural evolution.
The reason is that all of these conclusions use the effects of mutations as calculated using Equation~1, and the re-scaling simply acts as a constant multiplier on all mutational effects (e.g., a scale factor) when re-scaled preferences are converted to mutational effects.}
\end{quote}

Due to space constraints and the size of the logo plots, we cannot fit another main figure with the preferences prior to re-scaling, but such an image is now included in the Jupyter notebook in the manuscript's GitHub repository at \url{https://github.com/jbloomlab/Perth2009-DMS-Manuscript}.
The unscaled amino-acid preferences for each replicate are also provided in numerical form in Dataset S3.
}

6) The authors state that it there are more reports of escape from stalk antibodies for H3 than H1. This is simply not true. Actually, the first reference used as evidence for easy selection of H3 anti-stalk escape mutants (Chai et al) reported three escape mutants which a) did not disrupt binding completely and b) were crippled in vitro and in vivo. The authors should remove their statement as there is not systematic data to support this. 

\response{We have removed this sentence from the paper.}

7) The biggest issue with the manuscript is the experimental set up. The authors titrate their rescued virus using a TCID50 assay on MDCK cells. Based on the titer that they measure, they passage the virus from the initial rescue once onto fresh MDCK cells using a low MOI. After infection, the virus dilution stays on the cells and is then harvested without centrifugation followed by RNA extraction and sequencing. There are many issue with this process. 

\response{The reviewer has a number of questions and suggestions about the experimental design.
Some of these suggestions are things that were already implemented in our experiments and we simply failed to explain in adequate detail, while others are things that we have evidence are unimportant. 
Below we describe the issues one-by-one.

In addition, we have now included our \emph{full original lab notes} for the relevant experiments (\url{https://github.com/jbloomlab/Perth2009-DMS-Manuscript/tree/master/lab_notes}) so that readers can examine everything that we did in full detail.
This level of transparency goes well beyond that usually taken in scientific papers, and will enable readers to fully understand our reasoning and procedures for each detail of the experiments.
}

a) The TCID50 is likely underestimating the number of virus particles with HA genomic segments because not all of them (likely the minority) will be capable of inducting a full infectious cycle (but their presence might still be detected in the cell supernatants). 

\response{The reviewer is certainly correct that TCID50 assays underestimate the number of HA genomic segments, as it is well known that the number of physical influenza virions exceeds the number of infectious units by 10-100 fold.
This is why we used an extremely low MOI (0.0035) so that most cells are still infected by no more than one virion.
It is impossible to grow a viral stock that is completely free of defective particles.
However, in both our original rescue with helper virus and in the passage, we follow the steps suggested by Carolina Lopez and Scott Hensley (PMID 27047455) to minimize defective particles.
These include low MOI (0.0035 TCID50 per cell is quite low) and short growth times to avoid accumulation of defective particles (the helper-virus rescue was allowed to propagate for just 24 hours, and the low-MOI passage for just 48 hours).

In short, the presence of non-infectious particles is an inherent feature of all virology work.
The best that can be done is to use MOIs and growth times that ensure that they do not dominate the experiments, and we have done that.
}

b) The infectious inoculum is never washed off the infected cells meaning that any particles that are transferred from the rescue transfection are going to be detected during sequencing. This might even apply to naked RNPs released from dead cells. 

\response{
We apologize for not expanding on this detail in the Methods of the original paper.

In fact, like the reviewer, we initially thought that a media change to wash away the inoculum might be important. 
Therefore, in our experiments, we performed such a media change to remove the initial inoculum for biological replicates 1 and 2, and for one of the technical replicates (3-2) of biological replicate 3.
We did not perform such a change for the other technical replicate (3-1) of biological replicate 3.
It turns out that the media change makes no discernible difference, which is why we neglected to expand on this point in the original manuscript. 
We now have added the following explanatory text to the Methods:

\begin{quote}
\textsl{Three hours post-infection, we replaced the inoculum with fresh media for replicates 1, 2, and 3-2.
We did not perform a media change for replicate 3-1.
As can be seen in Figure~1D, the media change does not appear to have a substantial effect, as replicate 3-1 looks comparable to the other replicates.
}
\end{quote}
The full details of the media changes can be found in the lab notes we have included at We have also included in the GitHub repository a PDF of the lab notebook describing the rescue and passage of the mutant virus libraries at \url{https://github.com/jbloomlab/Perth2009-DMS-Manuscript/blob/master/lab_notes/Perth09_HA_Library_Rescue.pdf} (relevant details of the notebook can be found in pages 21-25).

We think that the reason that washing away the original inoculum has no discernible effect on the results it that the low concentration of virus in the inoculum makes a negligible contribution to the total number of virions that are present at the time of sequencing, since vastly more new virions are produced from the infected cells.
}

c) This would actually not be an issue, if the dilution from the rescue culture to the new culture would be large enough. But this is not the case. The authors passaged $9\times10^5$ TCID50 (and likely much more DIs/RNA) from the rescue cultures onto the new cells in a volume of 25 ml. This is, depending on experiment, between 10 and 1 ml of supernatant of the original rescue and leads to a 1:2.5 or 1:25 dilution respectively. 

\response{
We apologize if the original wording of our methods were unclear. 
For each mutant virus library, we passaged a total of $9\times10^5$ TCID50, but this was passaged across \emph{fifteen} 15-cm dishes \emph{each} of which received 25 ml.
Because we used an inoculum of 2.5 TCID50 / $\mu$l across these fifteen dishes, this corresponds to dilutions that range from 1:37 to 1:294.
The number of dishes used to passage each mutant virus library is now mentioned in the Methods; before it had to be calculated from the stated total number of cells and the cells per dish.

Also, we note that the supernatants from the helper virus rescues are collected at just 24 hours post-infection, which leads to lower viral titers (which is why we have to passage relatively large volumes), but also presumably much lower amounts of DIs / RNA.
This is probably the reason why (as described above) removing the inoculum did not discernibly affect the results.
}

d) The supernatant is never centrifuged before RNA extraction. Cell debris is not removed which is likely contributing significantly to RNA sequences detected in the supernatant. However, RNA associated with cell debris is not representative of RNA from fit, replicating viruses. 

\response{We apologize for failing to provide sufficiently detailed methods stating that we ultra-centrifuged \emph{clarified} the viral supernatant before RNA extraction.
The supernatant was clarified by centrifugation at 2000 $\times$ \textit{g} for five minutes to remove any cellular debris.
We have updated the Methods to now include this detail, and (as mentioned above) also now include our full lab notes so that all other details can be reviewed.
Specifically, see \url{https://github.com/jbloomlab/Perth2009-DMS-Manuscript/blob/master/lab_notes/Perth09_HA_Library_Rescue.pdf} (relevant details of the notebook can be found in pages 21-25).
}

The experiments should be repeated with at least one more passaging step, proper removal of inoculum and washing of the cells after infection. Centrifugation steps should also be included between the passages and before RNA extraction in the end. 

\response{For the reasons described above, it is not necessary to repeat the experiments.
We do not deny that there are slightly different ways that the experiments could be done, but in the end, the ``proof is in the pudding.'' 
In the manuscript, we show that the measurements made using the experimental parameters that we chose lead to useful data as judged by the following criterion:
\begin{itemize}
\item The results from fully independent experimental replicates are highly correlated (Fig.~1D), which would not be the case if the results were dominated by noise.
\item The experimental measurements can inform substitution models that describe influenza virus's evolution much better than standard models (Table~1).
\item The experimental measurements provide information that can help predict the fates of influenza virus lineages in nature (Fig.~5).
\end{itemize}
For any experimental study there are an almost infinite number of slight methodological variations that could be employed, and it is impossible to test them all.
Our experiments used reasonable parameter choices, were performed in full biological triplicate, and give meaningful / useful results with respect to the larger goals of the study.}

\subsubsection*{Minor points} 

1) Many abbreviations are not defined in the manuscript including MDCK-SIAT1, TMPRSS2, MOI, DMEM, FBS, LB. 

\response{We now explain these abbreviations in the manuscript.}

2) The authors state on page 5 that mutations detected in cell culture as deleterious rarely reach high frequency in nature. Could you please give a few examples? 

\response{We individually validated viral growth of a C52A mutant, which removes a Cys critical in forming a disulfide bond with Cys-277 and maintaining HA structure.
We confirmed that this mutant grows very poorly when rescued, and mutations at site 52 never appear in sequenced isolates.

As another example, mutation N170H fails to reach $>$1\% frequency in nature, and according to our dataset, is a mutation towards a highly unpreferred amino acid. 
}

3) The authors picked 31 random clones to evaluate the mutation rate per clone. These sequences should be made available. Did this set include viruses without a proper start codon (see above)? What was their growth phenotype in vitro? 

\response{We have now added the list of mutations from the 31 clones to Dataset S3.
Two of the clones had a mutation that removed the ATG start codon.
We did not test the growth of these clones. 
As explained in Figure~S2, these were plasmid variants chosen randomly to be Sanger sequenced to evaluate the mutation rate prior to creating virus libraries; they were not evaluated as viruses.
The new Figures S4 to S6 now show the growth phenotype of other selected mutants, one of which includes a start codon mutant.
}

\subsection*{Reviewer \#2 Comments} 

\subsubsection*{Comments:} 
The work presented in this manuscript is a thorough analysis of the tolerance of an H3 hemagglutinin to all possible single amino-acid mutations. H3 evolution is of particular relevance now in light of the most recent flu season where vaccine efficacy against H3N2 was estimated at only 25\%. The experimental setup presented in the manuscript is sound and there is strong correlation among the biological and technical replicates. The authors show that data from deep mutational scanning is better able to describe the evolution of H3 HA compared to conventional models. This is consistent with the results of their previous study using deep mutational scanning on the H1 HA of WSN. It is important to note that their approach is ``systematic and quantitative'' in capturing potential effects of mutations in the H3 HA on viral fitness in cell culture. The significance statement is accurate in describing the conclusions of the paper and is understandable to a general scientific audience. 

\response{We thank the reviewer for an accurate and fair summary of our work.}

\subsubsection*{Major points:} 

The authors themselves acknowledge the most important limitation of this study: that the experimental measure of viral growth in cell culture is not reflective of true fitness in nature. Most importantly, the work presented here does not capture host-induced pressures on the antigenic regions of the viral protein. This is evidenced by the discrepancy between the high mutational tolerance they calculated for the stalk region and the actual change in stalk amino acid sequence. A simple protein BLAST comparison between the amino acid sequences of HA from H3N2 viruses Perth09 and HK68 shows a 77\% conservation in head amino acids but a 94\% conservation in stalk amino acids. Additionally, according to figure 2, there is no observable reversion of the G78D mutation they introduced despite the dominance of G78 in the population as seen in figure S1. The authors recognize this in their work and do not attempt to exaggerate the significance of these findings. Still, it would be useful to differentiate between head and stalk mutations when depicting individual amino acid mutations in figure 4 and their frequency trajectories to highlight that the predominance of mutations that occur in nature are in the head. 

\response{We thank the reviewer for acknowledging our efforts to fairly present the considerable caveats associated with cell-culture measurements.

The suggestion to make a version of Fig.~4 that distinguished between head and stalk mutations is a good one.
It is difficult to also overlay the structural location (stalk versus head) on the current Fig.~4, since colors are already used to indicate mutational effects.
We have therefore created a new supplemental figure (Fig.~S11) that shows these data---as the reviewer suspected, this figure shows that head mutations reach high frequencies much more commonly than stalk mutations.

The reviewer is correct that G78D appears to be a site where the correspondence between selection in our experiments and frequency in nature is low.
This makes sense, as G78D is one of the two sites that we specifically identified as being a lab adaptation (see Figure~S1).
We now explicitly mention this fact in the Discussion as follows:

\begin{quote}
\textsl{Indeed, a vast amount of work in virology has chronicled the ways in which experiments can select for lab artifacts or fail to capture important pressures that are relevant in nature.
As an example, although we identified G78D as favorable for viral growth in cell culture, this mutation never fixes in nature. }
\end{quote}
} 

The data regarding greater mutational tolerance in the stalk compared to the head should be paired with data from deep mutational scanning utilizing stalk- and head-specific antibodies as a selection process as this group has previously done with WSN H1 (Doud et al., Nat Comm 2018, Doud et al., PLoS Path 2017) to see if tolerant sites are responsible for viral escape. 

\response{We agree that these would be very interesting experiments to perform.
However, such experiments would be far beyond the scope of our study.
As stated in the Abstract and Significance Statement, the main goal of our study is to perform deep mutational scanning of a recent H3 HA and to use the data to better predict H3N2 viral evolution in nature.
A large set of additional experiments comparing stalk- and head-specific antibodies in H3 and H1 HAs is clearly beyond the scope of this goal.
}

Additionally, the authors should directly show the discrepancies between this in vitro selection and the H3 HA's evolution in nature. This group has performed such an analysis in their previous deep mutational scanning work with the H1 HA of WSN (Doud and Bloom, Viruses 2016). Maximum likelihood phylogenetics can be used to estimate the difference between experimental amino acid preference and natural preference. One might expect that this would provide some insight into the nature of the bias introduced through cell culture. 

\response{This is a good suggestion and we have now added the results of this analysis in Figure S3.
}

The authors are upfront about the limited degree to which their calculated mutational effect and the maximum mutation frequency in nature correlate, describing it as ``modest.'' One of the major points of the paper is that conclusions from this type of analysis cannot be generalized when using data from strains that are too divergent. The authors should address the possibility that data from deep mutational scanning can be generalized to divergent strains if the study is done in a system that is more reflective of nature. 

\response{This is a good point, and we have added the following sentence to the Discussion: ``We suspect that performing similar experiments using more realistic and complex selections (e.g., ferrets or primary human airway cultures) might further improve their utility and possibly their generalizability to more divergent strains."}

The experimental approach of deep mutational scanning has been applied to viral proteins before, including HIV env, influenza H1 HA, and NP. The novelty in this paper is that it is the first time such an analysis has been performed for H3 specifically and that it demonstrates the (albeit limited) utility of deep mutational scanning in predicting the success of viral strains in nature. Ultimately, it is clear from the discussion that the authors see this study as an intermediate step, suggesting that their work is "an improvement over no information at all." While something is indeed better than nothing, in my opinion, this paper represents only an incremental step forward in the field of influenza evolutionary forecasting due to the aforementioned limitations, especially since this experiment and subsequent analysis could be performed using more stringent selection methods during viral passaging like treatment with human sera for data that is presumably more faithful to nature. 

\response{The reviewer is correct that additional experiments might lead to further improvements.
However, ours is the first paper to show that data from deep mutational scanning can actually help predict real evolutionary outcomes in nature---which is an important step forward that merits publication.

We are excited about the future directions that the reviewer suggests.
However, the current study is a complete and important contribution even if (as is almost invariably the case in science) future work can probably take things further.
Researchers have been performing functional experiments on influenza virus for decades, and have been trying to forecast the virus's evolution for decades---yet ours is the first study to successfully use such functional experiments to predict the success of viral strains in nature. 
We have gone to great pains to fairly discuss limitations and caveats and how things could be improved in the future, but that should not detract from the advances in the current paper.
}

\subsubsection*{Minor Points:} 

The correlation for prospective analysis (post-Perth) is smaller than the correlation for post-hoc analysis (pre-Perth). Since prospective analysis is more important when predicting evolutionary fates, it might be beneficial for the authors to comment on this difference. 

\response{
The reviewer is correct that on the complete data sets in the original submission, the correlation was slightly higher for the post-hoc (post-Perth) and prospective (pre-Perth) analysis.
But at the suggestion of Reviewer \#1, we have now also included a post-Perth analysis using \emph{only} HA sequences from unpassaged viral isolates (there are not enough unpassaged isolates from before 2009 to perform a similar unpassaged pre-Perth analysis).
As shown in the new Figure 5, the correlation for the unpassaged post-Perth is actually higher than either other set, suggesting that there may not a systematic trend for poorer performance in prospective analyses.
}

A number of issues interfere with the clarity of this manuscript, particularly when appealing to a broader scientific audience. While this is understandable given the nature of this study, several adjustments should be made. A clear description of mutational effect is never provided in the results, though it forms a crucial component of the data. As such, it is not really intuitive as to why the range from -10-0 should be considered "neutral." Additionally, mutational tolerance is quantified as Shannon entropy of the re-scaled amino-acid preferences at each site, but it is unclear how this entropy is calculated. The entire first half of the second results section describing how the experimentally informed codon substitution model they generate is better than existing models and the associated Table 1 is very difficult to understand. AIC is never explained. The significance of dN/dS being much less than 1 for conventional models should be explained. The purpose of modeling of ExpCM, site avg. is unclear. 

\response{We appreciate these suggestions from the reviewer and have made the following changes:
\begin{itemize}
\item We have now included a description of how we calculated mutational effects in the Results section.
\item We agree that a cutoff of -10 to 0 is arbitrary and removed the sentence ``Mutations that reach high frequencies generally have neutral or beneficial effects according to our experimental measurements." 
\item We included in the Methods a description of how we calculated Shannon entropy.
\item We now define AIC in the legend of Table 1.
\item We have now included an explanation of dN/dS $\ll$1 for conventional substitution models.
\item We now describe how the ``ExpCM, site avg.'' substitution model is a control that shows that the improvements of the ExpCM are due to the site-specific nature of the data, since the ``ExpCM, site avg.'' model averages data across sites.
\end{itemize}
}

In figure 4, it would be useful to show Perth09 on the tree and mention that closely related strains are excluded from analysis in the figure legend. As it stands, there is just an unlabeled cluster of viruses for which no individual amino acid mutations are characterized. It is also important for the authors to clearly state what frequency they are talking about. Specifically, in consideration of a particular mutation, what is the population in the denominator? 

\response{We have now labeled the Perth/2009 strain on the tree in Figure 4 and now mention in the caption that we have excluded closely related strains from the analysis.
We also now explain in the Methods that the frequencies are calculated in one month windows, and the denominator is the total number of sequences in that window.
}

The graph labels in figure 8B could be cleaner. The question marks after "absolutely conserved" and "clade-specific" are extraneous. Indeed, the graph labels themselves ("HA domain," "absolutely conserved? ""clade-specific?") are unnecessary. 

\response{We have now fixed the graph labels in Figure 8B as suggested by the reviewer.}




\end{document}  
